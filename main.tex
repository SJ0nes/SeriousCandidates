\documentclass{article}
\usepackage{graphicx} 

\title{Modelling Basic Quantum Mechanics Equations with Python:  }
\author{Samuel Jones, ...}
\date{April 2024}

\begin{document}

\maketitle

\begin{abstract}
This paper introduces QuanTom : a python library that helps solve A-level physics questions regarding Quantum Mechanics(QM). Within this document, we will explain how our library resolves these types of questions and describe the advanced capability of our library.
\end{abstract}

\section{Introduction}
Quantum mechanics emerged in the early 20th century when scientists attempted to understand the behaviour of lights and electrons. This then developed into theories of wave duality and the photoelectric effect etc. (QM) helps us understand the behaviour of particles like electrons and photons. This concept is used in many practical fields for example: within technology such as transistors, lasers and LEDs; within medicine for MRI and PET scans

This library includes formulas from [2]
\begin{itemize}
    \item Wavespeed Equation
    \item Photon Energy Equation
    \item Photoelectric effect Equation
    \item Kinetic Energy Equation
    \item De Broglie wave Equation
\end{itemize}
One of the main formulas used in QM is Kinetic Energy, we can use QuanTom to help solve these types of questions [1]. For example, if we were given the speed of the electron, we can then use our library to work out the KE of the particle (let's assume the velocity of the particle is 20$m/s$)
\begin{verbatim}
    >>> import Quantom
    >>> KE = Quantom.calculate_kinetic_energy(velocity=20)
    >>> KE
    1.822e-28
\end{verbatim}

\section{Statement of Need}
The QuanTom library aims to save time and aid understanding to A-level students within their Physics module as well as introduce coding to their everyday lives. Our library should be used in conjunction with other tools used within physics. For example; students should make sure that their units are all the same as the Python code. To our knowledge this the first library that integrates A-level quantum mechanics into code.

\section{Conclusion}
This project has given a description and insight into QuanTom: a library which helps students solve Quantum Mechanics (QM) questions
The library is well-documented and has been tested. 
\begin{itemize}
    \item python -m pip Quantom
    \item The source code is available at github...
\end{itemize}

\section{References}
\begin{enumerate}
    \item [1] Topic 4. 5: Quantum physics (no date) OCR A Physics A-level, Physicsandmathstutor.com. Available at: https://pmt.physicsandmathstutor.com/download/Physics/A-level/Notes/OCR-A/4-Electrons-Waves-Photons/Detailed/4.5.%20Quantum%20Physics.pdf (Accessed: April 27, 2024).
    \item [2] Quantum (no date) A Level Physics. Available at: https://www.alevelphysicsonline.com/quantum (Accessed: April 28, 2024).
\end{enumerate}
\end{document}
