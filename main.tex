\documentclass{article}
\usepackage{graphicx} 

\title{Modelling Basic Quantum Mechanics Equations with Python: Quantom}
\author{Aayla Haq, Anoj Rameshprabahar, Samuel Jones, Thomas McGarel}
\date{April 2024}


\begin{document}

\maketitle

\section{Summary}
Quantum physics is the branch of physics concerned with how matter and energy interact on the smallest scale. QuanTom is a Python library that provides students across the globe studying quantum physics or mechanics with relevant software to aid their calculations. This paper details the advanced capability of QuanTom and provides information concerning the library.

\section{Introduction to Quantum Physics}
Quantum physics is one of the most crucial branches of physics. It emerged in the early 20th century when scientists attempted to understand the most unique phenomena in nature by looking at the behaviour of the interaction between matter and energy alongside the activities that occur inside the atom. In the modern day, quantum physics has applications that are essential to everyday life. For example, this field was integral in the development of semiconductor technology that is used in smartphones, computers, and other electronic devices that we rely on daily. Other prominent examples include transistors, lasers, light-emitting diodes, MRI machines and PET scans.

\section{The Software}
The functions that can be utilised in this library consist of: the wave speed equation, the photon energy equation, the photoelectric equation, the kinetic energy equation and the De Broglie wavelength equation. The library 

To demonstrate the utilisation of a QuanTom functions, kinetic energy can be used. For example, if provided with the velocity of a particle, the library can be used to calculate the kinetic energy of the particle. 

let's assume the velocity of the particle is $20ms^2$
\begin{verbatim}
    >>> import Quantom
    >>> KE = Quantom.calculate_kinetic_energy(velocity=20)
    >>> KE
    1.822e-28
\end{verbatim}

\section{Statement of Need}
Quantum physics is a diverse field and covers many fundamental concepts that are taught to students from the beginning of their secondary school education to those who pursue STEM degrees at university. These concepts consist of wave-particle duality, the photoelectric effect, quantum entanglement, quantum superposition, quantum tunnelling and mechanics. 

The QuanTom library covers the fundamental equations necessary in the education of physics systems and aims to aid students with lengthy calculations. In 2021 in the UK alone, 36,698 students took  A-Level Physics, 88,244 took A-Level Mathematics and 14,562 took A-Level Further Maths (1) and all 3 subjects cover equations which are available in the QuanTom library.  The QuanTom library provides a useful tool to students and teachers by providing a quick and efficient method of calculation. The library provides clear answers and reduces the chances of error (compared to typing in the numbers by hand on a calculator). This library will support students’ understanding of quantum calculations and will incorporate the usage of Python into their education. 

Aside from education, the QuanTom library can be used by industry professionals. For example, quantum technology is a newly emerging area of research in many technology industries. QuanTom provides professionals working in fields such as quantum computing, quantum simulation, quantum communications and quantum sensing with an accessible and essential library. 


\section{State of the Field}
QuanTom is a library that doesn't depend on any other libraries and can be easily integrated.

The most mature software available on quantum physics is QuTip (2). QuTip is a library that is “designed for simulating the open quantum dynamics of systems” 2, however, the library is not suitable for students and teachers as the functionality is too advanced for GCSE and A-Level. 
Another adversary to QuanTom is QEngine3 which is a library for “simulating and controlling ultracold quantum systems”3. QEngine is written in C++ which makes it less relevant to researchers and students as Python is the growing language in research and is the most easily accessible and simple-to-use language. Although all 3 libraries are based on quantum physics, there is no overlapping content between QuanTom and QEngine and QuTip, making it a unique software.

\section{Conclusion}
QuanTom is designed to be a toolbox for students and teachers working with quantum physics calculations. The software is accessible through the GitHub repository linked below where a thorough tutorial and guide have been written to provide detailed instructions on how the software is to be used.  

\begin{itemize}
    \item The source code is available at GitHub: https://github.com/SJ0nes/SeriousCandidates
\end{itemize}

\section{References}
\begin{enumerate}
    \item  UK Government . (2022, May 17). “Entries and Results - A level and AS by subject and student characteristics (single academic year)” from “A level and other 16 to 18 results”, Permanent data table. Explore-Education-Statistics.service.gov.uk. https://explore-education-statistics.service.gov.uk/data-tables/permalink/4000ba98-bbe6-4b65-b26c-508c06b578fb.
    \item J. R. Johansson, P.D. Nation, and F. Nori, “QuTiP 2: A Python framework for the dynamics of open quantum systems”, Comp. Phys. Comm. 184, 1234 (2013). https://qutip.org/ 
    \item Sørensen, J. J., Jensen, J., Heinzel, T., and Sherson, J. (2018). QEngine: An open-source C++ Library for Quantum Optimal Control of Ultracold Atoms [C++ library]. https://arxiv.org/abs/1807.11731

\end{enumerate}

\begin{small}
The following text is a recommended reference on quantum mechanics:
\begin{verbatim}
>Piccirillo, L. (2023). Introduction to the Maths and Physics of Quantum Mechanics. CRC Press.    
\end{verbatim}


The following textbook is a recommended reference on quantum physics:
\begin{verbatim}
>O’neill, M. (2015). OCR AS/A level physics A. Student book 1. Pearson.    
\end{verbatim}
\end{small}



\end{document}
