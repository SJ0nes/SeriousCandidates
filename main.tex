\documentclass{article}
\usepackage[utf8]{inputenc}
\usepackage{amsmath}

\title{Modelling Magic with Python: Ertai}
\author{Vince Knight}
\date{2021-04-16}

\begin{document}

\maketitle

\begin{abstract}
This paper introduces Ertai: a python library that allows for the mathematical modelling of a popular Collectible Card Game (CCG): Magic the Gathering (MtG). As well as describing the high level functionality of this library, a specific use case will be given that allows for the calculation of a ‘Mana Curve’ which is of strategic interest to players of MtG.
\end{abstract}

\section{Introduction}
Magic the Gathering (MtG) is a Collectible Card Game (CCG) first released in 1993 and is arguably the first and most popular CCG. The game has a number of strategic elements of gameplay:
\begin{enumerate}
    \item Card selection: from one's collection deciding which cards to use to form a deck.
    \item In play strategy: during the play against an opponent, the large number of card combinations allow for many strategic possibilities.
    \item Card trading: by design cards have varying value and are often the subject of trades.
\end{enumerate}
The ertai library aims to allow for the simulation of certain aspects of game play as applied to the first of these: how to select cards in a specific way. At its core MtG has a dynamic of casting spells (playing cards) which requires a player to spend “Mana”. This resource can be of various types, that correspond to the colours of Magic in MtG: black, white, blue, red or green. 

\begin{verbatim}
import ertai
mana_cost = ertai.Mana("Blue", "Red", "Red")
mana_pool = ertai.Mana("Blue", "Blue", "Blue", "Red", "Red", "Black")
mana_cost <= mana_pool
True
mana_pool - mana_cost
2 Blue Mana, 1 Black Mana
\end{verbatim}

\section{Statement of need}
This type of analysis is not novel, for example a number of other libraries that can be used to simulate plays of the game are described, for example: Magarena, which is an open source library built in Java. The ertai library is the first Python library (to the authors knowledge) which allows it to be readily used in conjunction with other scientific tools. Furthermore one particular goal of ertai is to be specifically translatable to mathematical models of MtG.

\section{Conclusion}
This paper has given a description of ertai: a library with a goal of allowing for the mathematical modelling of MtG. The current capabilities of the library are limited but as it is written in Python, the object oriented nature of the library can be used to make the library extendable to other aspects of MtG. Every ability of a Magic card can be added as a method on the ertai.Card class.

\bibliographystyle{plain}
\bibliography{references}

\end{document}
